\documentclass[ngerman,twocolumn,DIV21,a4,10pt]{scrartcl}
\usepackage{babel}
\usepackage{amsmath}
\usepackage{units}
\usepackage{nomencl}
\makenomenclature
\begin{document}
\title{Kommentare zum Praktikumsversuch: Heterogenes chemisches Gleichgewicht}
\author{Martin Kielhorn}
\maketitle
\section{Ziel des Versuches:}
Es ist das Massenwirkungsgesetz auf das Zersetzungsgleichgewicht eines
Nickel-Hexammin-Komplexes anzuwenden. Aus der Temperaturabhängigkeit
der Gleichgewichtskonstanten sind die Standardreaktionsenergie, die
Standardreaktionsenthalpie sowie die Standardreaktionsentropie zu
ermittteln.
\section{Begriffe}

\printnomenclature
\subsection{Reaktionenergie}
\subsection{Reaktionenthalpie}
\subsection{Freie Reaktionenergie}
\subsection{Freie Reaktionenthalpie}
\subsection{Thermodynamisches Gleichgewicht}
\newcommand{\dr}{\Delta_\textrm{R}}
\newcommand{\eqref}{1}{(\ref{#1})}
\newcommand{\dc}{{}^\circ\textrm{C}}
\begin{itemize}
\item konstante Temperatur, konstanter Druck: $\dr G = 0$
\item konstantes Volumen $\dr F = 0$
\end{itemize}
\section{Theoretische Grundlagen}
Bei einigen Ammoniak-Additionsverbindungen stellt sich umkehrbar und
relativ schnell ein Zersetzungsgleichgewicht ein. Beispielsweise kann
festes Ni-Hexamminchlorid mit festem Diamminchlorid und Ammoniak in
der Gasphase im Gleichgewicht stehen:

\begin{align}
\label{eqn:hex}
\underbrace{[Ni(NH_3)_6]Cl_2}_s \rightarrow 
\underbrace{[Ni(NH_3)_2]Cl_2}_s + \underbrace{4 NH_3}_g
\end{align}

Da feste (s) und gasförmige (g) Phasen beteiligt sind, spricht man von
einem ``heterogenen Gleichgewicht''.  Das Massenwirkungsgesetz für
Gl.\eqref{eqn:hex} lautet:
\begin{align}
K(T) = \frac{a_\textrm{Diammin} \cdot a^4_{NH_3}}{a_\textrm{Hexammin}}
\end{align}
wobei $a$ die \nomenclature{Aktivit\"at}{bla} Aktivit\"aten der
jeweiligen Stoffe und $K(T)$ die temperaturabhängige
Gleichgewichtskonstante sind. Solange die beteiligten Feststoffe in
reinen Phasen nebeneinander vorliegen (und nicht etwa als feste
L\"osung der einen Substanz in der anderen, wobei der Molenbruch jeder
Substanz kleiner als 1 w\"are), ist die Molenbruchaktivit\"at gleich
1.  Wird als Standardzustand des gasförmigen $NH_3$ das ideale Gas
unter einem Druck von $p = \unit[10^5]{Pa} = \unit[1]{Bar}$ gewählt,
ist die Aktivit\"at
\begin{align}
a_{NH_3}=\frac{p_{NH_3}}{p^\theta},
\end{align}
so dass das Massenwirkungsgesetz die einfache Form
\begin{align}
K(T)=\left(\frac{p_{NH_3}}{p^\theta}\right)^4
\end{align}
annimmt. Die Temperaturabhängigkeit der Gleichgewichtskonstanten
ergibt sich direkt aus der Messung des $NH_3-$Gleichgewichtsdruckes
als Funktion der Temperatur.

Der Versuch wird bei konstantem Volumen durchgeführt. Für die
Gleichgewichtskonstante $K$ gilt dann folgende Beziehung (Lehrbücher
der Physikal. Chemie):
\begin{align}
\label{eqn:frei}
\dr F^\theta = -RT\ln K.
\end{align}
Unter Verwendung der Gibbs-Helmholtz-Gleichung
\begin{align}
\label{eqn:frei2}
\dr F^\theta = \dr U^\theta - T \cdot \dr S^\theta
\end{align}
ergibt sich
\begin{align}
\ln K = -\frac{\dr U^\theta}{R}\frac{1}{T} + \frac{\dr S^\theta}{R}.
\end{align}
Unter der (im benutzten engen Temperaturintervall in guter Näherung
erfüllten) Voraussetzung, dass RU und R S temperaturunabhängig sind,
hängt $\ln K$ also linear von $1/T$ ab. Wenn $K$ als Funktion der
Temperatur gemessen und $\ln K$ als Funktion $1/T$ graphisch
dargestellt wird, kann man aus dem Anstieg bzw. Ordinatenabschnitt der
sich ergebenden Geraden $\dr U^\theta$ und $\dr S^\theta$
erhalten. Dazu ist es zweckmäßig, einige Umformungen
durchzuführen. (Im folgenden werden die Indizes $NH_3$ weggelassen.)

Wir ersetzen $\ln K$ mit $\ln K=4\ln(p/p^\theta)$ und Normaldruck
$p^\theta = \unit[10^5]{Pa}$ und erhalten eine lineare Gleichung in $y
= \ln(p/p^\theta)$ und $x=1/T$:
\begin{align}
  \label{eqn:fit}
  \underbrace{\ln \left(\frac{p(T)}{p^\theta}\right)}_y &= 
  \underbrace{\frac{\dr S^\theta}{4R}}_A \underbrace{-\frac{\dr U^\theta}{4R}}_B
  \frac{1}{T} \\
  y(x) &= A+B x
\end{align}
Aus den dimensionslosen Fit-Parametern $A$ und $B$ können die
Standardreaktionsenthalpie $\dr S^\theta$ und die \"Anderung der
inneren Energie bei Standardbedingungen $\dr U^\theta$ berechnet
werden.

Der lineare Fit ermoeglicht die Bestimmung der
Gleichgewichtskonstanten $K(T)$ f\"ur beliebige Temperaturen und auch
die freie Standardreaktionsenergie $\dr F^\theta$.

\paragraph{Note:}
A constant pressure in reaction \ref{eqn:hex} corresponds to the
equilibrium in the free enthalpy $\dr G=0$. In this case equations
\eqref{eqn:frei} and \eqref{eqn:frei2} are replaced by
\begin{align}
\dr G^\theta &= -RT \ln K \\
\ln K &= - \frac{\dr H^\theta}{R} \frac{1}{T} + \frac{\dr S^\theta}{R}
\end{align}
The enthalpy $\dr H^\theta=\dr U^\theta + p \dr V^\theta$ contains an
additional term that corresponds to the work that is necessary to
increase the volume of the gas. The ideal gas equations are a good
approximation for $NH_3$ and we can determine the work. According to
\eqref{eqn:hex}, \unit[1]{mol} of hexammin reacts to \unit[4]{mol} of
$NH_3$ gas.
\begin{align}
\dr V&= 4V_{NH_3}=4RT/p.
\end{align}
\section{Tasks}
\begin{enumerate}
\item Measure the disassociation pressure of hexammine $p$ for eight
  temperatures $T\in (90\ldots150^\circ\textrm{C})$.
\item Display the measured values and a linear fit according to
  equation \eqref{eqn:fit}.
\item Determine molar reaction enthalpy $\dr U^\theta$ and entropy
  $\dr S^\theta$ using the fit parameters $A$ and $B$.
\item Calculate the molar free reaction energy $\dr F^\theta(T)$, the
  equilibrium constant $K(T)$ and the molar reaction enthalpy $\dr
  H^\theta(T)$ for $T\in\{25\dc,75\dc,125\dc,175\dc\}$.
\end{enumerate}
\end{document}
